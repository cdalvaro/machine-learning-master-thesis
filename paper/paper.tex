\documentclass[11pt,a4paper,USenglish,twocolumn]{article}
\usepackage{unir_paper}


%---------------------------
%título del trabajo y autor
%---------------------------
\title{Machine Learning Tools for Open Cluster Characterization with Gaia DR2 Data}
\author{Carlos David Álvaro Yunta}
\date{24th of January, 2021}


%---------------------------
%marges
%---------------------------
%\usepackage[margin=1.9cm]{geometry}
%---------------------------
%---------------------------
%---------------------------
%---------------------------

\resumen{
The characterization and understanding of \emph{Open Clusters}
allow us to better understand properties and mechanisms about the Universe
such as stellar formation and the kind of regions where these events occur.
They also provide information about stellar processes and the evolution of the galactic disk.

In this paper, we develop a method to characterize open clusters by using
\emph{artificial intelligence} tools, such as clustering by \emph{K-Means}
and clustering based on \emph{Artificial Neural Networks} by implementing
the \emph{Deep Embedded Clustering} model.
We are using \emph{Gaia DR2 database} as data source for testing our models.

The developed method aims to improve current techniques of the state of the art.
We achieve improvements not only in terms of \emph{computational efficiency},
with lower computational requirements, but in \emph{usability},
reducing the number of hyperparameters to obtain a good characterization of the analyzed clusters.

Our method achieves good results,
becoming even better in some cases when the results are compared with current techniques.
}

\palabrasclave{
    characterization,
    data analysis,
    deep embedded clustering,
    gaia,
    machine learning,
    open cluster
}

\begin{document}
\twocolumn[
\begin{@twocolumnfalse}
\maketitle
\end{@twocolumnfalse}
]

%\renewcommand{\listfigurename}{Índice de Ilustraciones}
%\renewcommand{\listtablename}{Índice de Tablas}
%\renewcommand{\contentsname}{Índice de Contenidos}
%\renewcommand{\figurename}{Figura}
%\renewcommand{\tablename}{Tabla}
%\twocolumn


%\frontmatter
%\tableofcontents
%\listoffigures
%\listoftables

\section{INTRODUCTION}

\section{STATE OF THE ART}

Estudio a fondo el dominio de aplicación, citando numerosas referencias.
Debe aportar un buen resumen del conocimiento que ya existe en el campo de los problemas habituales identificados.
\section{AIMS AND METHOD}

Objetivo general, objetivos específicos y metodología de trabajo aplicada.

\section{CONTRIBUTION}

\section{RESULTS}
\subsection{Evaluación 1}

\subsection{Evaluación 2 }
En la Tabla \ref{tab_1}
\begin{table}\label{tab_1}
\caption{Unidades de las propiedades magnéticas:}

\begin{tabular}{ccc}\hline\hline
Símbolo & Cantidad & Conversión\\
\hline
$\Phi$ & flujo magnético & $1$Mx $\rightarrow 10^{-8}V\cdot s$\\
... &...&...\\
\hline\hline
\end{tabular}
\end{table}

\section{RESULTS DISCUSSION}

Tras la presentación objetiva de los resultados, querrás aportar una discusión de los mismos.

\section{CONCLUSIONS}

Resumen de las contribuciones del trabajo, en el que relaciones las contribuciones y los resultados obtenidos con los objetivos que habías planteado para el trabajo, discutiendo hasta qué punto has conseguido resolver los objetivos planteados.
Finalmente, hablar de líneas de trabajo futuro que podrían aportar valor añadido al TFM realizado. La sección debería señalar las perspectivas de futuro que abre el trabajo desarrollado para el campo de estudio definido. En el fondo, debes justificar de qué modo puede emplearse la aportación que has desarrollado y en qué campos.

\appendix
\section{APPENDIX}
Apéndices, en caso de ser necesario.

\renewcommand{\refname}{REFERENCES}
\bibliographystyle{unsrt}
\bibliography{references}

\end{document}
