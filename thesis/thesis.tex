\documentclass[11pt, a4paper, english]{book}
\usepackage{unir}

\setlength{\headheight}{15pt}

%---------------------------
% Título del trabajo y autor
%---------------------------
\title{Open Clusters Detection in Gaia DR2 Using ML Algorithms}
\author{Carlos David Álvaro Yunta}
\date{13 de Septiembre de 2020}
\director{César Augusto Guzmán Álvarez}
\nombreciudad{Madrid, Spain}

%---------------------------
%marges
%---------------------------
%\usepackage[margin=1.9cm]{geometry}
%---------------------------
%---------------------------
%---------------------------
%---------------------------
\begin{document}
%\renewcommand{\listfigurename}{Índice de Ilustraciones}
%\renewcommand{\listtablename}{Índice de Tablas}
%\renewcommand{\contentsname}{Índice de Contenidos}
%\renewcommand{\figurename}{Figura}
%\renewcommand{\tablename}{Tabla}

\maketitle

\frontmatter
\tableofcontents
\listoffigures
\listoftables

\chapter{Abstract}
{\bf Nota:} En este apartado se introducirá un breve resumen en español del trabajo realizado (extensión máxima: 150 palabras). Este resumen debe incluir el objetivo o propósito de la investigación, la metodología, los resultados y las conclusiones.


{\bf Key words.} open clusters detection --- machine learning --- gaia dr2 --- data analysis

\chapter{Resumen}
{\bf Nota:} En este apartado se introducirá un breve resumen en español del trabajo realizado (extensión máxima: 150 palabras). Este resumen debe incluir el objetivo o propósito de la investigación, la metodología, los resultados y las conclusiones.


{\bf Palabras Clave:} detección de cúmulos abiertos --- inteligencia artificial --- gaia dr2 --- análisis de datos

\mainmatter
\chapter{Introduction}

Hunting for Open Clusters\cite{castro2020hunting}

\chapter{Context and State of the Art}

\chapter{Identificación de Requisitos}

\chapter{Aims}

\chapter{Desarrollo del trabajo}

\chapter{Conclusiones y Trabajo Futuro}

\chapter*{Acknowledgement}
\addcontentsline{toc}{chapter}{Acknowledgement}

This work has made use of data from the European Space Agency (ESA) mission
{\it Gaia} (\url{https://www.cosmos.esa.int/gaia}), processed by the {\it Gaia}
Data Processing and Analysis Consortium (DPAC,
\url{https://www.cosmos.esa.int/web/gaia/dpac/consortium}). Funding for the DPAC
has been provided by national institutions, in particular the institutions
participating in the {\it Gaia} Multilateral Agreement.

\bibliography{references}
\bibliographystyle{plain}
\addcontentsline{toc}{chapter}{Bibliography}

\appendix
\chapter{Appendix}
Atención, deberá generar un pdf con la plantilla de artículo y añadirla como anexo utilizando includepdf.

%\includepdf[pages=-]{anexo.pdf} # TODO: Descomentar cuando el artículo esté hecho
\end{document}
