\documentclass[11pt, a4paper, english]{book}
\usepackage{unir}

\setlength{\headheight}{15pt}

%---------------------------
% Título del trabajo y autor
%---------------------------
\title{Open Clusters Detection in Gaia DR2 Using ML Algorithms}
\author{Carlos David Álvaro Yunta}
\date{13 de Septiembre de 2020}
\director{César Augusto Guzmán Álvarez}
\nombreciudad{Madrid, Spain}

%---------------------------
%marges
%---------------------------
%\usepackage[margin=1.9cm]{geometry}
%---------------------------
%---------------------------
%---------------------------
%---------------------------
\begin{document}
%\renewcommand{\listfigurename}{Índice de Ilustraciones}
%\renewcommand{\listtablename}{Índice de Tablas}
%\renewcommand{\contentsname}{Índice de Contenidos}
%\renewcommand{\figurename}{Figura}
%\renewcommand{\tablename}{Tabla}

\maketitle

\frontmatter
\tableofcontents
\listoffigures
\listoftables

\chapter{Abstract}
{\bf Nota:} En este apartado se introducirá un breve resumen en español del trabajo realizado (extensión máxima: 150 palabras). Este resumen debe incluir el objetivo o propósito de la investigación, la metodología, los resultados y las conclusiones.


{\bf Key words.} open clusters detection --- machine learning --- gaia dr2 --- data analysis

\chapter{Resumen}
{\bf Nota:} En este apartado se introducirá un breve resumen en español del trabajo realizado (extensión máxima: 150 palabras). Este resumen debe incluir el objetivo o propósito de la investigación, la metodología, los resultados y las conclusiones.


{\bf Palabras Clave:} detección de cúmulos abiertos --- inteligencia artificial --- gaia dr2 --- análisis de datos

\mainmatter
\chapter{Introduction}

Stellar open clusters (OC) are groups of stars gravitationally bound originated from a single molecular gas cloud.
Thus they share the same chemical composition and age, and they have similar relative positions and proper motion.
These astronomical objects are of fundamental importance to understand the spiral structure,
the dynamics and the chemical evolution of our galaxy.

Although most stars in the Milky Way are presented isolated, it is considered that most of them (or even all)
are formed in clustered environments and spend a period of time gravitationally bound with their siblings embedded
in their original molecular cloud.
\cite[Clarke et al. 2000]{clarke2000theformationofstellarclusters} \cite[Portegies Zwart et al. 2010]{portegies2010young}
The evolution of these systems tends to sparse them in a few million years by interacting gravitationally with other systems.
Galactic tidal forces and mechanisms that involves the gas loss driven by stellar feedback are other causes of disruption.
\cite[Brinkmann et al. 2017]{brinkmann2017bound}
Nevertheless, a small fraction of these systems will survive in the initial state and persist bound in bigger timescales.

Young OC allow us to research stars formation regions and improve our understanding about the mechanisms that create those stars.
On the other side, older OC give us information about stellar processes and how the galactic disk evolves.
Some highly disturbed orbits could also provide evidence of recent merge events and accretion traces from outside the galaxy.
\cite[Cantat-Gaudin et al. 2016]{cantat2016abundances}

The study of OC has been pushed forward thanks to the huge and precise dataset from the Gaia mission \cite[Gaia DR2]{gaia2018gaia},
available since 2018. This dataset has helped to review already known open clusters and to find new ones.

\chapter{Context and State of the Art}

\chapter{Identificación de Requisitos}

\chapter{Aims}

\chapter{Desarrollo del trabajo}

\chapter{Conclusiones y Trabajo Futuro}

\chapter*{Acknowledgement}
\addcontentsline{toc}{chapter}{Acknowledgement}

This work has made use of data from the European Space Agency (ESA) mission
{\it Gaia} (\url{https://www.cosmos.esa.int/gaia}), processed by the {\it Gaia}
Data Processing and Analysis Consortium (DPAC,
\url{https://www.cosmos.esa.int/web/gaia/dpac/consortium}). Funding for the DPAC
has been provided by national institutions, in particular the institutions
participating in the {\it Gaia} Multilateral Agreement.

\bibliography{references}
\bibliographystyle{plain}
\addcontentsline{toc}{chapter}{Bibliography}

\appendix
\chapter{Appendix}
Atención, deberá generar un pdf con la plantilla de artículo y añadirla como anexo utilizando includepdf.

%\includepdf[pages=-]{anexo.pdf} # TODO: Descomentar cuando el artículo esté hecho
\end{document}
